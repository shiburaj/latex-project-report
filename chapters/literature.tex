\chapter{Review of Literature}
Present a critical appraisal of the previous work published in the literature pertaining to the topic of the investigation.

\section{Review of Paper 1: Credit Card Fraud Detection Using Random Forest}

\begin{table}[h]
\centering
\caption{Summary of Paper 1}
\label{tab:paper1}
\begin{tabular}{ll}
\toprule
\textbf{Attribute} & \textbf{Details} \\
\midrule
Authors & Smith et al. (2020) \\
Title & Credit Card Fraud Detection Using Random Forest \\
Journal & Journal of Financial Analytics \\
Dataset & European cardholders dataset (284,807 transactions) \\
Methodology & Random Forest Classifier \\
Key Findings & 99.8\% accuracy, 0.92 F1-score for fraud class \\
Limitations & High computational cost for real-time processing \\
\bottomrule
\end{tabular}
\end{table}

\subsection*{Methodology and Approach}
Smith et al. (2020) employed a Random Forest algorithm for credit card fraud detection, focusing on handling class imbalance through stratified sampling. Their approach involved feature engineering on transaction time, amount, and historical patterns.

\subsection*{Strengths and Contributions}
\begin{itemize}
\item Comprehensive feature engineering approach
\item Effective handling of class imbalance
\item High accuracy rates on benchmark dataset
\item Robust against overfitting
\end{itemize}

\subsection*{Limitations and Gaps}
\begin{itemize}
\item Computational intensity limits real-time application
\item Limited exploration of deep learning alternatives
\item Dataset restricted to European transactions only
\end{itemize}

% Second Paper
\section{Review of Paper 2: Deep Learning for Anomaly Detection in Financial Transactions}

\begin{table}[h]
\centering
\caption{Summary of Paper 2}
\label{tab:paper2}
\begin{tabular}{ll}
\toprule
\textbf{Attribute} & \textbf{Details} \\
\midrule
Authors & Johnson and Brown (2021) \\
Title & Deep Learning for Anomaly Detection in Financial Transactions \\
Conference & IEEE International Conference on Data Science \\
Dataset & Synthetic financial dataset (500,000 transactions) \\
Methodology & Autoencoder Neural Network \\
Key Findings & 99.5\% accuracy, better recall than traditional methods \\
Limitations & High false positive rate in imbalanced scenarios \\
\bottomrule
\end{tabular}
\end{table}

\subsection*{Methodology and Approach}
Johnson and Brown (2021) proposed an autoencoder-based approach for unsupervised anomaly detection. Their model learned normal transaction patterns and flagged deviations as potential fraud, eliminating the need for labeled fraud data.

\subsection*{Strengths and Contributions}
\begin{itemize}
\item Unsupervised approach reduces dependency on labeled data
\item Effective in detecting novel fraud patterns
\item Scalable architecture for large datasets
\item Good generalization across different transaction types
\end{itemize}

\subsection*{Limitations and Gaps}
\begin{itemize}
\item High false positive rates in practical applications
\item Computational complexity during training phase
\item Limited interpretability of detection results
\end{itemize}